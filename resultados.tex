\chapter{Testes empíricos e análise dos resultados}

TEXTO AQUI.

\section{Roteiro e fundamentação do teste proposto}

Este trabalho propõe como teste de validação para técnica de classificação
descrita nos capítulos anteriores um esquema simples de comparação de
expectativas, haverá um grupo de imagens previamente classificado por agentes
humanos, e outra classificação, para este mesmo grupo de imagens, gerada pela
técnica aqui proposta, deste modo, será possível efetuar a comparação entre a
expectativa, isto é, as classes que idealmente devem ser geradas, representadas
pela classificação humana das imagens, com as classes fornecidas pela técnica
baseada nas redes de Kohonen. As imagens utilizadas no teste são de tal modo
que sua classificação humana é auto-evidente, e deste modo, como será observado
na Seção \ref{sec:conjunto_de_imagens}, dificilmente serão alvo de alguma
contestação.

O objetivo do teste é avaliar a quantidade e a pertinência das classes, ou seja,
se o mesmo número de classes aparece em ambos as classificações e se as classes
geradas pelo método proposto possuem correspondência com as classes definidas
pelos agentes humanos, mais especificamente, se as imagens estão agrupadas pelo
método baseado nas redes de Kohonen do mesmo modo, ou de modo muito semelhante,
como foram agrupadas pelos agentes humanos.

\section{Especificação do conjunto de imagens e das classes de controle}
\label{sec:conjunto_de_imagens}

O conjunto de imagens utilizada para a execução do teste é o
\textit{Columbia Object Image Library} (COIL-100), um conjunto de 100
objetos em 7200 poses
diferentes. Destes 100, foram escolhidas 91 imagens de diferentes objetos para
compor o teste. Foram identificadas 15 classes para estas imagens, cada classe possui
imagens em poses com variações de translação, rotação e escala para o mesmo tipo
de objeto, afim de confrontar as premissas da Seção \ref{sec:momentos_desc} no
contexto do processo de classificação.

As classes identificadas foram nomeadas e apresentam as seguintes imagens:

\begin{table}[H]
  \centering
  \caption{Grupo A (animais de brinquedo).}
  \tabulinesep =_0.5em^0.5em
  \everyrow{\tabucline[0.4pt]-}
  \begin{tabu}{|cccccc|}
    \includegraphics[width=0.1\textwidth,height=0.1\textwidth]{imagens/coil_100/animais_brinquedos/obj14__0.png} &
    \includegraphics[width=0.1\textwidth,height=0.1\textwidth]{imagens/coil_100/animais_brinquedos/obj14__0_1.png} &
    \includegraphics[width=0.1\textwidth,height=0.1\textwidth]{imagens/coil_100/animais_brinquedos/obj17__0.png} &
    \includegraphics[width=0.1\textwidth,height=0.1\textwidth]{imagens/coil_100/animais_brinquedos/obj17__0_1.png} &
    \includegraphics[width=0.1\textwidth,height=0.1\textwidth]{imagens/coil_100/animais_brinquedos/obj20__0.png} &
    \includegraphics[width=0.1\textwidth,height=0.1\textwidth]{imagens/coil_100/animais_brinquedos/obj28__275.png}
    \\
    \scriptsize{obj01.jpg} & \scriptsize{obj02.jpg} & \scriptsize{obj03.jpg} &
    \scriptsize{obj04.jpg} & \scriptsize{obj05.jpg} & \scriptsize{obj06.jpg}
    \\
    \includegraphics[width=0.1\textwidth,height=0.1\textwidth]{imagens/coil_100/animais_brinquedos/obj28__275_1.png} &
    \includegraphics[width=0.1\linewidth,height=0.1\linewidth]{imagens/coil_100/animais_brinquedos/obj48__265.png} &
    \includegraphics[width=0.1\linewidth,height=0.1\linewidth]{imagens/coil_100/animais_brinquedos/obj52__0.png} &
    \includegraphics[width=0.1\linewidth,height=0.1\linewidth]{imagens/coil_100/animais_brinquedos/obj52__0_1.png} &
    \includegraphics[width=0.1\linewidth,height=0.1\linewidth]{imagens/coil_100/animais_brinquedos/obj74__0.png} &
    \\
    \scriptsize{obj07.jpg} & \scriptsize{obj08.jpg} & \scriptsize{obj09.jpg} &
    \scriptsize{obj10.jpg} & \scriptsize{obj11.jpg} &
  \end{tabu}
\end{table}

\begin{table}[H]
  \centering
  \caption{Grupo B (barquinhos de brinquedo).}
  \tabulinesep =_0.5em^0.5em
  \everyrow{\tabucline[0.4pt]-}
  \begin{tabu}{|ccc|}
    \includegraphics[width=0.1\textwidth,height=0.1\textwidth]{imagens/coil_100/barquinhos_brinquedos/obj3__0.png} &
    \includegraphics[width=0.1\textwidth,height=0.1\textwidth]{imagens/coil_100/barquinhos_brinquedos/obj38__0.png} &
    \includegraphics[width=0.1\textwidth,height=0.1\textwidth]{imagens/coil_100/barquinhos_brinquedos/obj38__0_1.png}
    \\
    \scriptsize{obj12.jpg} & \scriptsize{obj13.jpg} & \scriptsize{obj14.jpg}
    \\
    \includegraphics[width=0.1\textwidth,height=0.1\textwidth]{imagens/coil_100/barquinhos_brinquedos/obj42__0.png} &
    \includegraphics[width=0.1\textwidth,height=0.1\textwidth]{imagens/coil_100/barquinhos_brinquedos/obj42__0_1.png} &
    \includegraphics[width=0.1\textwidth,height=0.1\textwidth]{imagens/coil_100/barquinhos_brinquedos/obj78__0.png}
    \\
    \scriptsize{obj15.jpg} & \scriptsize{obj16.jpg} & \scriptsize{obj17.jpg}
  \end{tabu}
\end{table}

\begin{table}[H]
  \centering
  \caption{Grupo C (boias).}
  \tabulinesep =_0.5em^0.5em
  \everyrow{\tabucline[0.4pt]-}
  \begin{tabu}{|ccc|}
    \includegraphics[width=0.1\textwidth,height=0.1\textwidth]{imagens/coil_100/boias/obj47__0.png} &
    \includegraphics[width=0.1\textwidth,height=0.1\textwidth]{imagens/coil_100/boias/obj94__0.png} &
    \includegraphics[width=0.1\textwidth,height=0.1\textwidth]{imagens/coil_100/boias/obj94__0_1.png}
    \\
    \scriptsize{obj18.jpg} & \scriptsize{obj19.jpg} & \scriptsize{obj20.jpg}
  \end{tabu}
\end{table}

\begin{table}[H]
  \centering
  \caption{Grupo D (caixas).}
  \tabulinesep =_0.5em^0.5em
  \everyrow{\tabucline[0.4pt]-}
  \begin{tabu}{|ccccccc|}
    \includegraphics[width=0.1\textwidth,height=0.1\textwidth]{imagens/coil_100/caixas/obj1__35.png} &
    \includegraphics[width=0.1\textwidth,height=0.1\textwidth]{imagens/coil_100/caixas/obj1__35_1.png} &
    \includegraphics[width=0.1\textwidth,height=0.1\textwidth]{imagens/coil_100/caixas/obj31__45.png} &
    \includegraphics[width=0.1\textwidth,height=0.1\textwidth]{imagens/coil_100/caixas/obj46__45.png} &
    \includegraphics[width=0.1\textwidth,height=0.1\textwidth]{imagens/coil_100/caixas/obj46__45_1.png} &
    \includegraphics[width=0.1\textwidth,height=0.1\textwidth]{imagens/coil_100/caixas/obj54__55.png} &
    \includegraphics[width=0.1\textwidth,height=0.1\textwidth]{imagens/coil_100/caixas/obj67__50.png}
    \\
    \scriptsize{obj21.jpg} & \scriptsize{obj22.jpg} & \scriptsize{obj23.jpg} &
    \scriptsize{obj24.jpg} & \scriptsize{obj25.jpg} & \scriptsize{obj26.jpg} &
    \scriptsize{obj27.jpg}
    \\
    \includegraphics[width=0.1\textwidth,height=0.1\textwidth]{imagens/coil_100/caixas/obj79__45.png} &
    \includegraphics[width=0.1\textwidth,height=0.1\textwidth]{imagens/coil_100/caixas/obj84__45.png} &
    \includegraphics[width=0.1\textwidth,height=0.1\textwidth]{imagens/coil_100/caixas/obj84__45_1.png} &
    \includegraphics[width=0.1\textwidth,height=0.1\textwidth]{imagens/coil_100/caixas/obj96__45.png} &
    \includegraphics[width=0.1\textwidth,height=0.1\textwidth]{imagens/coil_100/caixas/obj98__55.png} &
    \includegraphics[width=0.1\textwidth,height=0.1\textwidth]{imagens/coil_100/caixas/obj98__55_1.png} &
    \\
    \scriptsize{obj28.jpg} & \scriptsize{obj29.jpg} & \scriptsize{obj30.jpg} &
    \scriptsize{obj31.jpg} & \scriptsize{obj32.jpg} & \scriptsize{obj33.jpg} &
  \end{tabu}
\end{table}

\begin{table}[H]
  \centering
  \caption{Grupo E (carrinhos de brinquedo).}
  \tabulinesep =_0.5em^0.5em
  \everyrow{\tabucline[0.4pt]-}
  \begin{tabu}{|ccccccc|}
    \includegraphics[width=0.1\textwidth,height=0.1\textwidth]{imagens/coil_100/carrinhos_brinquedos/obj6__0.png} &
    \includegraphics[width=0.1\textwidth,height=0.1\textwidth]{imagens/coil_100/carrinhos_brinquedos/obj8__0.png} &
    \includegraphics[width=0.1\textwidth,height=0.1\textwidth]{imagens/coil_100/carrinhos_brinquedos/obj15__0.png} &
    \includegraphics[width=0.1\textwidth,height=0.1\textwidth]{imagens/coil_100/carrinhos_brinquedos/obj19__0.png} &
    \includegraphics[width=0.1\textwidth,height=0.1\textwidth]{imagens/coil_100/carrinhos_brinquedos/obj19__0_1.png} &
    \includegraphics[width=0.1\textwidth,height=0.1\textwidth]{imagens/coil_100/carrinhos_brinquedos/obj23__0.png} &
    \includegraphics[width=0.1\textwidth,height=0.1\textwidth]{imagens/coil_100/carrinhos_brinquedos/obj27__0.png}
    \\
    \scriptsize{obj34.jpg} & \scriptsize{obj35.jpg} & \scriptsize{obj36.jpg} &
    \scriptsize{obj37.jpg} & \scriptsize{obj38.jpg} & \scriptsize{obj39.jpg} &
    \scriptsize{obj40.jpg}
    \\
    \includegraphics[width=0.1\textwidth,height=0.1\textwidth]{imagens/coil_100/carrinhos_brinquedos/obj27__0_1.png} &
    \includegraphics[width=0.1\textwidth,height=0.1\textwidth]{imagens/coil_100/carrinhos_brinquedos/obj37__0.png} &
    \includegraphics[width=0.1\textwidth,height=0.1\textwidth]{imagens/coil_100/carrinhos_brinquedos/obj69__0.png} &
    \includegraphics[width=0.1\textwidth,height=0.1\textwidth]{imagens/coil_100/carrinhos_brinquedos/obj69__0_1.png} &
    \includegraphics[width=0.1\textwidth,height=0.1\textwidth]{imagens/coil_100/carrinhos_brinquedos/obj76__0.png} &
    \includegraphics[width=0.1\textwidth,height=0.1\textwidth]{imagens/coil_100/carrinhos_brinquedos/obj91__0.png} &
    \includegraphics[width=0.1\textwidth,height=0.1\textwidth]{imagens/coil_100/carrinhos_brinquedos/obj100__0.png}
    \\
    \scriptsize{obj41.jpg} & \scriptsize{obj42.jpg} & \scriptsize{obj43.jpg} &
    \scriptsize{obj44.jpg} & \scriptsize{obj45.jpg} & \scriptsize{obj46.jpg} &
    \scriptsize{obj47.jpg}
  \end{tabu}
\end{table}

\begin{table}[H]
  \centering
  \caption{Grupo F (chícaras).}
  \tabulinesep =_0.5em^0.5em
  \everyrow{\tabucline[0.4pt]-}
  \begin{tabu}{|cccccc|}
    \includegraphics[width=0.1\textwidth,height=0.1\textwidth]{imagens/coil_100/chicaras/obj10__0.png} &
    \includegraphics[width=0.1\textwidth,height=0.1\textwidth]{imagens/coil_100/chicaras/obj11__0.png} &
    \includegraphics[width=0.1\textwidth,height=0.1\textwidth]{imagens/coil_100/chicaras/obj16__0.png} &
    \includegraphics[width=0.1\textwidth,height=0.1\textwidth]{imagens/coil_100/chicaras/obj16__0_1.png} &
    \includegraphics[width=0.1\textwidth,height=0.1\textwidth]{imagens/coil_100/chicaras/obj43__0.png} &
    \includegraphics[width=0.1\textwidth,height=0.1\textwidth]{imagens/coil_100/chicaras/obj43__0_1.png}
    \\
    \scriptsize{obj48.jpg} & \scriptsize{obj49.jpg} & \scriptsize{obj50.jpg} &
    \scriptsize{obj51.jpg} & \scriptsize{obj52.jpg} & \scriptsize{obj53.jpg}
    \\
    \includegraphics[width=0.1\textwidth,height=0.1\textwidth]{imagens/coil_100/chicaras/obj45__0.png} &
    \includegraphics[width=0.1\textwidth,height=0.1\textwidth]{imagens/coil_100/chicaras/obj59__0.png} &
    \includegraphics[width=0.1\textwidth,height=0.1\textwidth]{imagens/coil_100/chicaras/obj59__0_1.png} &
    \includegraphics[width=0.1\textwidth,height=0.1\textwidth]{imagens/coil_100/chicaras/obj81__0.png} &
    \includegraphics[width=0.1\textwidth,height=0.1\textwidth]{imagens/coil_100/chicaras/obj89__0.png} &
    \includegraphics[width=0.1\textwidth,height=0.1\textwidth]{imagens/coil_100/chicaras/obj97__0.png}
    \\
    \scriptsize{obj54.jpg} & \scriptsize{obj55.jpg} & \scriptsize{obj56.jpg} &
    \scriptsize{obj57.jpg} & \scriptsize{obj58.jpg} & \scriptsize{obj59.jpg}
  \end{tabu}
\end{table}

\begin{table}[H]
  \centering
  \caption{Grupo G (embalagens cilíndricas).}
  \tabulinesep =_0.5em^0.5em
  \everyrow{\tabucline[0.4pt]-}
  \begin{tabu}{|cccccc|}
    \includegraphics[width=0.1\textwidth,height=0.1\textwidth]{imagens/coil_100/embalagens_cilindricas/obj7__0.png} &
    \includegraphics[width=0.1\textwidth,height=0.1\textwidth]{imagens/coil_100/embalagens_cilindricas/obj26__0.png} &
    \includegraphics[width=0.1\textwidth,height=0.1\textwidth]{imagens/coil_100/embalagens_cilindricas/obj26__0_1.png} &
    \includegraphics[width=0.1\textwidth,height=0.1\textwidth]{imagens/coil_100/embalagens_cilindricas/obj29__0.png} &
    \includegraphics[width=0.1\textwidth,height=0.1\textwidth]{imagens/coil_100/embalagens_cilindricas/obj32__0.png} &
    \includegraphics[width=0.1\textwidth,height=0.1\textwidth]{imagens/coil_100/embalagens_cilindricas/obj49__0.png}
    \\
    \scriptsize{obj60.jpg} & \scriptsize{obj61.jpg} & \scriptsize{obj62.jpg} &
    \scriptsize{obj63.jpg} & \scriptsize{obj64.jpg} & \scriptsize{obj65.jpg}
    \\
    \includegraphics[width=0.1\textwidth,height=0.1\textwidth]{imagens/coil_100/embalagens_cilindricas/obj62__80.png} &
    \includegraphics[width=0.1\textwidth,height=0.1\textwidth]{imagens/coil_100/embalagens_cilindricas/obj71__0.png} &
    \includegraphics[width=0.1\textwidth,height=0.1\textwidth]{imagens/coil_100/embalagens_cilindricas/obj87__0.png} &
    \includegraphics[width=0.1\textwidth,height=0.1\textwidth]{imagens/coil_100/embalagens_cilindricas/obj93__0.png} &
    \includegraphics[width=0.1\textwidth,height=0.1\textwidth]{imagens/coil_100/embalagens_cilindricas/obj93__0_1.png} &
    \includegraphics[width=0.1\textwidth,height=0.1\textwidth]{imagens/coil_100/embalagens_cilindricas/obj99__0.png}
    \\
    \scriptsize{obj66.jpg} & \scriptsize{obj67.jpg} & \scriptsize{obj68.jpg} &
    \scriptsize{obj69.jpg} & \scriptsize{obj70.jpg} & \scriptsize{obj71.jpg}
  \end{tabu}
\end{table}

\begin{table}[H]
  \centering
  \caption{Grupo H (embalagens retangulares).}
  \tabulinesep =_0.5em^0.5em
  \everyrow{\tabucline[0.4pt]-}
  \begin{tabu}{|cccc|}
    \includegraphics[width=0.1\textwidth,height=0.1\textwidth]{imagens/coil_100/embalagens_retangulares/obj9__30.png} &
    \includegraphics[width=0.1\textwidth,height=0.1\textwidth]{imagens/coil_100/embalagens_retangulares/obj22__0.png} &
    \includegraphics[width=0.1\textwidth,height=0.1\textwidth]{imagens/coil_100/embalagens_retangulares/obj22__0_1.png} &
    \includegraphics[width=0.1\textwidth,height=0.1\textwidth]{imagens/coil_100/embalagens_retangulares/obj39__55.png}
    \\
    \scriptsize{obj72.jpg} & \scriptsize{obj73.jpg} & \scriptsize{obj74.jpg} &
    \scriptsize{obj75.jpg}
    \\
    \includegraphics[width=0.1\textwidth,height=0.1\textwidth]{imagens/coil_100/embalagens_retangulares/obj55__0.png} &
    \includegraphics[width=0.1\textwidth,height=0.1\textwidth]{imagens/coil_100/embalagens_retangulares/obj65__50.png} &
    \includegraphics[width=0.1\textwidth,height=0.1\textwidth]{imagens/coil_100/embalagens_retangulares/obj65__50_1.png} &
    \includegraphics[width=0.1\textwidth,height=0.1\textwidth]{imagens/coil_100/embalagens_retangulares/obj90__0.png}
    \\
    \scriptsize{obj76.jpg} & \scriptsize{obj77.jpg} & \scriptsize{obj78.jpg} &
    \scriptsize{obj79.jpg}
  \end{tabu}
\end{table}

\begin{table}[H]
  \centering
  \caption{Grupo I (embalagens com tampa).}
  \tabulinesep =_0.5em^0.5em
  \everyrow{\tabucline[0.4pt]-}
  \begin{tabu}{|cccccc|}
    \includegraphics[width=0.1\textwidth,height=0.1\textwidth]{imagens/coil_100/embalagens_tampas/obj5__0.png} &
    \includegraphics[width=0.1\textwidth,height=0.1\textwidth]{imagens/coil_100/embalagens_tampas/obj5__0_1.png} &
    \includegraphics[width=0.1\textwidth,height=0.1\textwidth]{imagens/coil_100/embalagens_tampas/obj13__40.png} &
    \includegraphics[width=0.1\textwidth,height=0.1\textwidth]{imagens/coil_100/embalagens_tampas/obj24__0.png} &
    \includegraphics[width=0.1\textwidth,height=0.1\textwidth]{imagens/coil_100/embalagens_tampas/obj33__0.png} &
    \includegraphics[width=0.1\textwidth,height=0.1\textwidth]{imagens/coil_100/embalagens_tampas/obj33__0_1.png}
    \\
    \scriptsize{obj80.jpg} & \scriptsize{obj81.jpg} & \scriptsize{obj82.jpg} &
    \scriptsize{obj83.jpg} & \scriptsize{obj84.jpg} & \scriptsize{obj85.jpg}
    \\
    \includegraphics[width=0.1\textwidth,height=0.1\textwidth]{imagens/coil_100/embalagens_tampas/obj50__0.png} &
    \includegraphics[width=0.1\textwidth,height=0.1\textwidth]{imagens/coil_100/embalagens_tampas/obj61__0.png} &
    \includegraphics[width=0.1\textwidth,height=0.1\textwidth]{imagens/coil_100/embalagens_tampas/obj64__0.png} &
    \includegraphics[width=0.1\textwidth,height=0.1\textwidth]{imagens/coil_100/embalagens_tampas/obj88__0.png} &
    \includegraphics[width=0.1\textwidth,height=0.1\textwidth]{imagens/coil_100/embalagens_tampas/obj92__0.png} &
    \includegraphics[width=0.1\textwidth,height=0.1\textwidth]{imagens/coil_100/embalagens_tampas/obj92__0_1.png}
    \\
    \scriptsize{obj86.jpg} & \scriptsize{obj87.jpg} & \scriptsize{obj88.jpg} &
    \scriptsize{obj89.jpg} & \scriptsize{obj90.jpg} & \scriptsize{obj91.jpg}
  \end{tabu}
\end{table}

\begin{table}[H]
  \centering
  \caption{Grupo J (ganchos).}
  \tabulinesep =_0.5em^0.5em
  \everyrow{\tabucline[0.4pt]-}
  \begin{tabu}{|ccc|}
    \includegraphics[width=0.1\textwidth,height=0.1\textwidth]{imagens/coil_100/ganchos/obj36__0.png} &
    \includegraphics[width=0.1\textwidth,height=0.1\textwidth]{imagens/coil_100/ganchos/obj85__0.png} &
    \includegraphics[width=0.1\textwidth,height=0.1\textwidth]{imagens/coil_100/ganchos/obj85__0_1.png}
    \\
    \scriptsize{obj92.jpg} & \scriptsize{obj93.jpg} & \scriptsize{obj94.jpg}
  \end{tabu}
\end{table}

\begin{table}[H]
  \centering
  \caption{Grupo L (lanches).}
  \tabulinesep =_0.5em^0.5em
  \everyrow{\tabucline[0.4pt]-}
  \begin{tabu}{|ccc|}
    \includegraphics[width=0.1\textwidth,height=0.1\textwidth]{imagens/coil_100/lanches/obj53__0.png} &
    \includegraphics[width=0.1\textwidth,height=0.1\textwidth]{imagens/coil_100/lanches/obj53__0_1.png} &
    \includegraphics[width=0.1\textwidth,height=0.1\textwidth]{imagens/coil_100/lanches/obj73__0.png}
    \\
    \scriptsize{obj95.jpg} & \scriptsize{obj96.jpg} & \scriptsize{obj97.jpg}
  \end{tabu}
\end{table}

\begin{table}[H]
  \centering
  \caption{Grupo M (legumes e frutas).}
  \tabulinesep =_0.5em^0.5em
  \everyrow{\tabucline[0.4pt]-}
  \begin{tabu}{|cccccc|}
    \includegraphics[width=0.1\textwidth,height=0.1\textwidth]{imagens/coil_100/legumes_frutas/obj2__0.png} &
    \includegraphics[width=0.1\textwidth,height=0.1\textwidth]{imagens/coil_100/legumes_frutas/obj2__0_1.png} &
    \includegraphics[width=0.1\textwidth,height=0.1\textwidth]{imagens/coil_100/legumes_frutas/obj4__0.png} &
    \includegraphics[width=0.1\textwidth,height=0.1\textwidth]{imagens/coil_100/legumes_frutas/obj63__0.png} &
    \includegraphics[width=0.1\textwidth,height=0.1\textwidth]{imagens/coil_100/legumes_frutas/obj63__0_1.png} &
    \includegraphics[width=0.1\textwidth,height=0.1\textwidth]{imagens/coil_100/legumes_frutas/obj75__0.png}
    \\
    \scriptsize{obj98.jpg} & \scriptsize{obj99.jpg} & \scriptsize{obj100.jpg} &
    \scriptsize{obj101.jpg} & \scriptsize{obj102.jpg} & \scriptsize{obj103.jpg}
    \\
    \includegraphics[width=0.1\textwidth,height=0.1\textwidth]{imagens/coil_100/legumes_frutas/obj75__0_1.png} &
    \includegraphics[width=0.1\linewidth,height=0.1\linewidth]{imagens/coil_100/legumes_frutas/obj82__0.png} &
    \includegraphics[width=0.1\linewidth,height=0.1\linewidth]{imagens/coil_100/legumes_frutas/obj83__0.png} &
    \includegraphics[width=0.1\linewidth,height=0.1\linewidth]{imagens/coil_100/legumes_frutas/obj83__0_1.png} &
    \includegraphics[width=0.1\linewidth,height=0.1\linewidth]{imagens/coil_100/legumes_frutas/obj86__0.png} &
    \\
    \scriptsize{obj104.jpg} & \scriptsize{obj105.jpg} & \scriptsize{obj106.jpg} &
    \scriptsize{obj107.jpg} & \scriptsize{obj108.jpg} &
  \end{tabu}
\end{table}

\begin{table}[H]
  \centering
  \caption{Grupo N (objetos de madeira).}
  \tabulinesep =_0.5em^0.5em
  \everyrow{\tabucline[0.4pt]-}
  \begin{tabu}{|cccc|}
    \includegraphics[width=0.1\textwidth,height=0.1\textwidth]{imagens/coil_100/objetos_madeira/obj12__0.png} &
    \includegraphics[width=0.1\textwidth,height=0.1\textwidth]{imagens/coil_100/objetos_madeira/obj12__0_1.png} &
    \includegraphics[width=0.1\textwidth,height=0.1\textwidth]{imagens/coil_100/objetos_madeira/obj41__0.png} &
    \includegraphics[width=0.1\textwidth,height=0.1\textwidth]{imagens/coil_100/objetos_madeira/obj41__0_1.png}
    \\
    \scriptsize{obj109.jpg} & \scriptsize{obj110.jpg} & \scriptsize{obj111.jpg} &
    \scriptsize{obj112.jpg}
    \\
    \includegraphics[width=0.1\textwidth,height=0.1\textwidth]{imagens/coil_100/objetos_madeira/obj51__0.png} &
    \includegraphics[width=0.1\textwidth,height=0.1\textwidth]{imagens/coil_100/objetos_madeira/obj51__0_1.png} &
    \includegraphics[width=0.1\textwidth,height=0.1\textwidth]{imagens/coil_100/objetos_madeira/obj77__0.png} &
    \includegraphics[width=0.1\linewidth,height=0.1\linewidth]{imagens/coil_100/objetos_madeira/obj80__0.png}
    \\
    \scriptsize{obj113.jpg} & \scriptsize{obj114.jpg} & \scriptsize{obj115.jpg} &
    \scriptsize{obj116.jpg}
  \end{tabu}
\end{table}

\begin{table}[H]
  \centering
  \caption{Grupo O (potes).}
  \tabulinesep =_0.5em^0.5em
  \everyrow{\tabucline[0.4pt]-}
  \begin{tabu}{|ccccc|}
    \includegraphics[width=0.1\textwidth,height=0.1\textwidth]{imagens/coil_100/potes/obj70__0.png} &
    \includegraphics[width=0.1\textwidth,height=0.1\textwidth]{imagens/coil_100/potes/obj70__0_1.png} &
    \includegraphics[width=0.1\textwidth,height=0.1\textwidth]{imagens/coil_100/potes/obj72__0.png} &
    \includegraphics[width=0.1\textwidth,height=0.1\textwidth]{imagens/coil_100/potes/obj72__0_1.png} &
    \includegraphics[width=0.1\textwidth,height=0.1\textwidth]{imagens/coil_100/potes/obj95__0.png}
    \\
    \scriptsize{obj117.jpg} & \scriptsize{obj118.jpg} & \scriptsize{obj119.jpg} &
    \scriptsize{obj120.jpg} & \scriptsize{obj121.jpg}
  \end{tabu}
\end{table}

\begin{table}[H]
  \centering
  \caption{Grupo P (vasos).}
  \tabulinesep =_0.5em^0.5em
  \everyrow{\tabucline[0.4pt]-}
  \begin{tabu}{|cccc|}
    \includegraphics[width=0.1\textwidth,height=0.1\textwidth]{imagens/coil_100/vasos/obj18__0.png} &
    \includegraphics[width=0.1\textwidth,height=0.1\textwidth]{imagens/coil_100/vasos/obj18__0_1.png} &
    \includegraphics[width=0.1\textwidth,height=0.1\textwidth]{imagens/coil_100/vasos/obj25__0.png} &
    \includegraphics[width=0.1\textwidth,height=0.1\textwidth]{imagens/coil_100/vasos/obj30__0.png}
    \\
    \scriptsize{obj122.jpg} & \scriptsize{obj123.jpg} & \scriptsize{obj124.jpg} &
    \scriptsize{obj125.jpg}
    \\
    \includegraphics[width=0.1\textwidth,height=0.1\textwidth]{imagens/coil_100/vasos/obj30__0_1.png} &
    \includegraphics[width=0.1\textwidth,height=0.1\textwidth]{imagens/coil_100/vasos/obj56__0.png} &
    \includegraphics[width=0.1\textwidth,height=0.1\textwidth]{imagens/coil_100/vasos/obj56__0_1.png} &
    \includegraphics[width=0.1\textwidth,height=0.1\textwidth]{imagens/coil_100/vasos/obj58__0.png}
    \\
    \scriptsize{obj126.jpg} & \scriptsize{obj127.jpg} & \scriptsize{obj128.jpg} &
    \scriptsize{obj129.jpg}
  \end{tabu}
\end{table}

\section{Consideraçoes sobre a implementação e a plataforma de execução}

Os testes foram executados em um computador com 64 bits de endereçamento,
processador Intel Core i5-2450M CPU 2.50GHz, 2 núcleos e 4 processadores
lógicos. O sistema operacional utilizado foi a distribuição Linux Ubuntu 12.04
Precise Pangolin LTS de 64 bits. Os códigos que implementam o método proposto
foram todos escritos em ANSI C, e compilados utilizando GCC no modo para gerar
código com maior desempenho.

A rede foi configurada para possuir um mapa de 80$ \times $40 neurônios, o valor
inicial para a largura efetiva de vizinhança ($ \sigma_0 $) foi definido para
$ 0,4 $, a taxa de aprendizagem inicial ($ \eta_0 $) para $ 0,2 $,
e a constante de tempo ($ \tau_l $) para $ 0,2 $.

\section{Treinamento da rede, convergência do erro e variação da U-matriz}

Tendo as imagens, as etapas de (a) binarização, (b) extração dos momentos e (c)
normalização foram executadas, as entradas foram organizadas de tal modo que
não houvesse nenhuma sequência grande de imagens pertencente a uma mesma classe,
para evitar o enviesamento do aprendizado. Após o treinamento, tendo sido
capturado o erro global da rede em cata etapa, a sua convergência ocorreu
como demostrado na Figura \ref{fig:error}:

\begin{figure}[H]
  \begin{center}
    \includegraphics[height=4cm]{imagens/error.pdf}
  \end{center}
  \caption{ Variação do erro durante as 533 iterações necessárias para completar
    o treinamento da rede.}
  \label{fig:error}
\end{figure}

A variação da U-matriz em seis momentos específicos pode ser avaliada na
Figura \ref{fig:som_var}:

\begin{figure}[H]
  \centering

  \begin{subfigure}{0.3\textwidth}
    \includegraphics[width=\textwidth]{imagens/som6_aux.jpg}
    \caption{Iteração 1}
    \label{fig:som1}
  \end{subfigure}~
  \begin{subfigure}{0.3\textwidth}
    \includegraphics[width=\textwidth]{imagens/som5_aux.jpg}
    \caption{Iteração 5}
    \label{fig:som2}
  \end{subfigure}~
  \begin{subfigure}{0.3\textwidth}
    \includegraphics[width=\textwidth]{imagens/som4_aux.jpg}
    \caption{Iteração 10}
    \label{fig:som3}
  \end{subfigure}~

  \begin{subfigure}{0.3\textwidth}
    \includegraphics[width=\textwidth]{imagens/som3_aux.jpg}
    \caption{Iteração 15}
    \label{fig:som4}
  \end{subfigure}~
  \begin{subfigure}{0.3\textwidth}
    \includegraphics[width=\textwidth]{imagens/som2_aux.jpg}
    \caption{Iteração 20}
    \label{fig:som5}
  \end{subfigure}~
  \begin{subfigure}{0.3\textwidth}
    \includegraphics[width=\textwidth]{imagens/som1_aux.jpg}
    \caption{Iteração 533}
    \label{fig:som6}
  \end{subfigure}

  \caption{Variação da U-matriz ao longo do treinamento da rede.}
  \label{fig:som_var}
\end{figure}

Como o erro converge rapidamente para um valor pequeno, e permanesse ocilando
não muito acima deste valor até que efetivamente zera. Então as amostras dos
estados intermediários da U-matriz foram tiradas de iteraçoes não muito longe
do início do processo, e uma do fim para demostrar o resultado final.

É nítido a presença de vales e fronteiras, e por conseguinte, a presença de
grupos identificados.

\section{Tempo de execução}

TEXTO AQUI

\section{Disposição das imagens e rotulação das imagens no mapa}

A U-matriz resultante do treinamento da rede permite a definição das regiões.
As imagens agora podem ser classificadas, no caso, posicionadas na grade e
deste modo mapeadas em uma das regiões, concluındo assim todo o processo.

A Figura \ref{fig:grupos_final} abaixo representa graficamente os grupos
definidos pela transformada de \textit{watershed} sobre a U-matriz resultante:

\begin{figure}[H]
  \begin{center}
    \includegraphics[height=12cm]{imagens/grupos_final.pdf}
  \end{center}
  \caption{ Variação do erro durante as 533 iterações necessárias para completar
    o treinamento da rede.}
  \label{fig:grupos_final}
\end{figure}

Foram 15 grupos identificados no total, cada um deles apresenta as seguintes
imagens:

\section{Avaliação geral dos resultados}

TEXTO AQUI.
