\chapter{Introdução}

A partir do momento que um grande volume de dados passou a ser armazenado em
meios digitais, técnicas automáticas para classificação ou agrupamento
(\textit{clustering}) foram desenvolvidas com o propósito de descobrir que tipo
de informação estes dados “escondem”, isto é, qual o tipo de ralação, muitas
vezes imperceptível aos usuários, os dados possuem entre si. Estas relações são,
em muitos casos, imprescindíveis na tomada de decisões, havendo assim um forte
interesse na criação e aperfeiçoar destas técnicas, seja para melhorar o tempo
de execução,  aumentar sua acurácia ou ainda ampliar a quantidade de informação
que pode ser inferida.

Inicialmente estas técnicas foram empregadas em dados puramente numéricos,
destacando-se informações bancárias, de geoprocessamento e informações sobre
contratantes de planos de seguro. Contudo, assim que imagens passaram a ser
processadas e armazenadas computacionalmente, surgiram também técnicas
destinadas a agrupá-las, com o mesmo objetivo, descobrir inter-relações entre
as imagens através do grupo para qual eram classificadas. Porém, com a
popularização da internet e das câmeras digitais diversos novos modos para
aquisição e compartilhamento de imagens surgiram, e com isso, as técnicas de
classificação adquiriram ou novo objetivo, passaram a ser utilizadas na
recuperação de imagens baseada em seu conteúdo visual, pois sabendo a que grupo
uma determinada imagem pertence, pode-se utilizar as imagens que estão no mesmo
grupo como resposta de uma consulta, ou no mínimo, como conjunto para uma busca
direcionada, evitando que a busca seja executada por todo conjunto de imagens.

Há, entretanto, uma dificuldade adicional ao se classificar imagens. De modo
geral, dados numéricos possuem um vasto ferramental teórico para definir
objetivamente o quanto são parecidos, isto é, o gral de semelhança entre eles.
Não é possível, porém, definir uma medida objetiva para o quanto semelhante são
duas imagens, isto porque os dados numéricos representam, por si só, uma
magnitude objetiva de uma determinada medida, as imagens no entanto, são
representações computacionais da visão humana, e como tal, sua representação,
aqui tomado como significado, está intimamente relacionada a forma como os
seres humanos, subjetivamente, a identificam. Duas pessoas podem divergir
significativamente quanto a semelhança entre duas imagens, pois cada uma pode
se apegar a uma interpretação cognitiva diferente, e nenhuma das duas estaria
errada quanto a essa medida.

A forma encontrada para contornar este problema foi extrair das imagens
determinados valores numéricos, chamado de descritores, e estes, ao invés das
imagens, são classificados. Existem diversos tipos de descritores, baseados em
diversas premissas sobre como se deve analisar a semelhança entre as imagens,
de modo que não são totalmente arbitrários e possibilitam que técnicas de
classificação convencionalmente utilizadas para outros tipos de dados também
sejam utilizadas com imagens.

Paralelo ao problema dos descritores, determinadas técnicas de classificação
impõe rígidas regras ao modo como as classes são determinadas, em particular,
precisam que o usuário defina a quantidade de classes que serão encontradas.
Isto pode fazer sentido para uma grande quantidade de classificações, como por
exemplo, para classificar imagens como “quentes” ou “frias”, para identificar
determinados gestos da mão ou expressões faciais, contudo, para os casos onde
não sabemos que tipo de relações desejamos encontrar, este tipo de abordagem é
ineficiente pois transfere ao usuário a obrigação de definir algo que, dado a
própria natureza do problema, ele não tem capacidade de saber, ou no máximo,
tem uma capacidade limitada. Contudo, técnicas mais sofisticadas não exigem este
tipo de informação, trazendo para seu próprio escopo a tarefa de inferir a
quantidade de classes de acordo com a natureza dos dados.

Por fim, classificar imagens significa agrupá-las segundo determinadas
características de similaridade; porém, o mapeamento deve ocorrer de tal forma
que cada classe forneça, essencialmente, as mesmas informações sobre as imagens
que a compõem. As técnicas de classificação podem fornecer um resumo conciso do
conteúdo das imagens, podendo ser utilizado para diferentes tarefas relacionadas
com a gestão de banco de dados de imagens.

\section{Tema}

No geral a classificação de imagens segue as abordagens convencionais para
classificação de dados,  a diferença consiste em que, as imagens, devido a sua
representação complexa, não são avaliadas diretamente pelos algoritmos, mas sim
um conjunto de descritores devidamente calculados. Estas técnicas de
classificação geralmente necessitam do número de classes que serão
identificados, as mais sofisticadas exigem apenas um número mínimo.

Este trabalho pretende modelar uma técnica de classificação de imagens centrada
nas redes neurais de Kohonen, que por isto não necessite da indicação explícita
do número de classes que serão encontradas, de modo que esta informação é um de
seus resultados e não um parâmetro necessário para sua aplicação.

A técnica é descrita pela apresentação formal das teorias que a compõe, bem
como suas devidas justificativas. É analisado as características, fundamentos
matemáticos e concepções de alto nível.

A técnica aqui apresentada não dispensa a necessidade de descritores, deste modo
os momentos invariantes de Hu são utilizados para este propósito. Uma análise de
todo o processo de extração de características é feito, o que
insere no contexto do trabalho conceitos de tratamento de imagens, em especial
binarização e o método de Otsu.

A rede de Kohonen é uma técnica emergente de aprendizado de máquina
direcionada a classificação de dados que se apoia no conceito
de mapas auto organizáveis para distribuir os elementos conforme seu gral de
similaridade. Associadas as redes de Kohonen estão a U-matriz e a transformada
de \textit{watershed}, que são utilizadas em conjunto para determinar as classes
para quais as imagens serão rotuladas.

\section{Objetivo geral}

Definir uma técnica de classificação de imagens onde os grupos são definidos de
acordo com formas (silhuetas) presentes nas imagens e que não haja a necessidade
do usuário informar a quantidade de classes a serem identificadas, ao contrário,
as classes devem ser um resultado emergente do processo.

\section{Objetivos específicos}

Associados ao objetivo geral, este trabalho também pretende:

\begin{itemize}

\item Investigar os métodos que compõe a técnica proposta, no caso,
  limiarização/binarização de imagens, método de Otsu para limiares globais, redes
  neurais de Kohonen, U-matriz e transformada de \textit{watershed};

\item Implementar a técnica em ANSI C;

\item Elaborar um conjunto de testes para validar os resultados da técnica
  proposta, identificando suas limitações caso existam.

\end{itemize}

\section{Justificativa}

Como já foi argumentado no início deste capítulo, as técnicas de agrupamento de
dados são importantes em diversas áreas, e em particular, a classificação de
imagens tem se destacado recentemente em virtude do forte crescimento da
internet, e com ela, o surgimento de uma nova modalidade de obtenção e
compartilhamento de imagens; e também, pela própria sofisticação de determinados
serviços de análise de imagens. Levando em consideração estes fatos, o
desenvolvimento e aperfeiçoamento de técnicas de classificação onde é necessário
o mínimo de interação humana, mais especificamente, onde as classes não precisem
ser definidas pelo usuário, suscita um grande interesse de pesquisa.

A ausência da necessidade de se definir a quantidade de classes não é apenas
uma questão de comodidade, mas também, em muitos casos, uma questão de
racionalidade metodológica, afinal, a definição do número de classes supõe certo
conhecimento sobre o grupo de imagens que se deseja classificar, contudo, na
maioria dos casos este conhecimento não existe, principalmente quando a
quantidade de imagens é muito grande, logo, uma técnica sem esta limitação é
imprescindível para estas situações.

\section{Organização do trabalho}

Esta monografia é dividida em 5 capítulos, onde inicialmente busca-se apresentar o
problema e o panorama geral onde o tema está situado, bem como o que motivou sua
investigação e resolução.

No Capítulo 2 uma breve introdução sobre a representação de imagens digitais é
feita, contudo, o objetivo do capítulo é descrever a extração de características
das imagens e os algorítimos envolvidos neste processo.

No Capítulo 3 é feito a formalização do processo de classificação propriamente
dito, assim como a contextualização e justificativa de todos os formalizamos
adotados na sua definição. O Capítulo se concentra especialmente nas redes
neurais de Kohonen e em como ela pode ser adotada na classificação das imagens
e na geração automática das classes.

Tanto no Capítulo 2 quanto no 3, ao final, são apresentados um resumo e um
diagrama para facilitar a compreensão do funcionamento global e da inter-relação
entre os componentes de cada etapa da classificação.

O Capítulo 4 descreve o roteiro de testes utilizados para validar a técnica,
assim como os resultados obtidos, também são apresentadas algumas analises para
dar suporte e interpretação aos resultados.

Por fim, o Capítulo 5 discorre sobre as conclusões acerca de tudo que foi
discutido, algumas avaliações e propostas de trabalhos futuros também são
indicadas neste capítulo.

O código fonte com alguns comentários adicionais sobre sua implementação pode
ser encontrado nos anexos do trabalho.
